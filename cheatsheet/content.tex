\begin{multicols*}{3}
\setlength{\premulticols}{1pt}
\setlength{\postmulticols}{1pt}
\setlength{\multicolsep}{1pt}
\setlength{\columnsep}{2pt}

\subsection{SQL Syntax}
\subsubsection{Top Level Queries}
\texttt{CREATE TABLE <table name>(<col name> <datatype> [row constraints], <table constraints>);}
\begin{itemize}[leftmargin=*]
    \item  Constraints can be \texttt{NOT NULL}, \texttt{PRIMARY KEY [AUTOINCREMENT]}, \texttt{UNIQUE}, \texttt{FOREIGN KEY ON <other table>(other col)}, \texttt{ CHECK <constraint>}
    \item  A primary key is a tuple of columns, non of which are \texttt{NULL} and uniquely identifies a row in the table.
    \item  A foreign key is a tuple of columns that reference a primary key in another table. If a value in the foreign key is \texttt{NULL}, the check is omitted.
    \item  \texttt{ON UPDATE/DELETE CASCADE/NO ACTION/SET DEFAULT/SET NULL} used to handle foreign keys.\\
\end{itemize}

\noindent\texttt{SELECT [DISTINCT] <columns> [AS <renamed column>] FROM <TABLE> WHERE <filter condition> GROUP BY <group condition> HAVING <group filter condition> ORDER BY [<columns> [ASC/DESC]]+ LIMIT <return limit>;}
\begin{itemize}[leftmargin=*]
    \item \texttt{DISTINCT} returns unique tuples of selected columns by removing subsequent duplicate rows. 
    \item \texttt{ORDER BY} sorts iteratively from the first given column. 
    \item \texttt{DISTINCT} and \texttt{ORDER BY} may not work together if the \texttt{ORDER BY} column is not in the selected columns as \texttt{DISTINCT} would not know which duplicate rows to remove.
    \item  If using \texttt{GROUP BY}, only values columns that are constant within each group can be selected.
    \item \texttt{EXTRACT(part FROM date)} can be used to retrieve specific components like month/minute from datetime types.\\
\end{itemize}

\noindent\texttt{INSERT INTO <table name> (<columns>) VALUES (<values>);} 
\begin{itemize}[leftmargin=*]
    \item \texttt{(<columns>)} can be omitted if inserting into every table column.
    \item Omitted columns are filled with \texttt{NULL} by default.\\
\end{itemize}

\noindent\texttt{UPDATE <table name> SET [<column> = <value>]+ WHERE <condition>;}\\

\noindent\texttt{DELETE FROM <table name> WHERE <condition>;}
\begin{itemize}[leftmargin=*]
    \item Omit where clause if deleting everything. \\
\end{itemize}

\noindent\textbf{Common Types}: INT, CHAR(1-255), NUMERIC, TEXT, VARCHAR(1-225)\\
\textbf{Operators} can be arithmetic, \texttt{||} is concatenation, \texttt{AS} is renaming, \texttt{CASE..WHEN..THEN..ELSE..END} is a conditional. Any operation on \texttt{NULL} results in \texttt{NULL}. \\
\textbf{Conditions} can be checked with \texttt{=,<,>,<=,>=,<>,IN,LIKE,BETWEEN ..AND.., IS [NOT] NULL} and joined with \texttt{AND,OR,NOT}. \\

\noindent A \textbf{Transaction} is a ordered list of operations that all commit or rollback. Constraint checks can be deferred to the end of a transaction.\\

\noindent A database is in a \textbf{consistent state} when it complies with \textbf{Integrity \& Structural Constraints}.\\
\textbf{Integrity Constraints} ensure data is of valid types and all conditions, inter and intra column, are satisfied. \\

\subsubsection{Aggregate and Algebraic Queries}
\textbf{Groups}: Logical groups of records with the same value for the specified fields. \texttt{HAVING} can only use aggregate functions or columns used for \texttt{GROUP BY}. \\
\textbf{Aggregate Operators} like \texttt{COUNT, SUM, MIN, MAX, AVG, STDDEV, ...} return a 1 value summary per \texttt{GROUP} for 1 column. \texttt{COUNT(*)} includes \texttt{NULL} values but all other operators ignore them. \texttt{(ALL ...)} is the default keyword but \texttt{DISTINCT} can also be used.\\

\noindent\textbf{Joins}: \texttt{t1 <join type> t2 [ON <conditions>]}. 
\begin{itemize}[leftmargin=*]
    \item              Usually join foreign keys to primary keys.
    \item \texttt{CROSS JOIN}: Cartesian product, all possible combinations of rows, no conditions required. Filter with the \texttt{WHERE} clause. Can also be written as \texttt{FROM t1, t2}.
    \item  \texttt{[INNER] JOIN}: Joins all rows where the condition is satisfied. No add expressiveness compared to \texttt{CROSS JOIN}, less preferred  as harder to read and less optimizable.
    \item \texttt{NATURAL JOIN}: Joins columns with the same name on the same values. No need for conditions
    \item \texttt{LEFT/RIGHT/FULL [OUTER] JOIN}: \texttt{INNER JOIN} and all values of the on the left/right/both sides with no matches on the other side which is padded with \texttt{NULL} values. Adds expressiveness. Can use \texttt{NATURAL} to join based on same column name condition.
    \item \texttt{ON} conditions in \texttt{OUTER JOIN} are not the same as in \texttt{INNER JOIN} or \texttt{WHERE} as non matches on the left/right/both sides are also included.
\end{itemize}

\noindent\textbf{Set Operations}: \texttt{UNION,INTERSECT \& EXCEPT} remove duplicates and only work on \underline{union-compatible} queries(same no. of columns, domain, and order). Adding \texttt{ALL} behind the operators prevents the removal of duplicates. Except adds expressiveness.\\

\subsubsection{Entity Relationship Diagrams}
\begin{tikzpicture}
    \node[entity] (ent) at (0,0){Entity};
    \node[relationship,aspect=3] (rel) at (2.5,0) {Relationship};
    % Aggregation
    \draw (4,-0.4) rectangle (6,0.4);
    \draw (4,0.0) -- (5,0.4) -- (6,0.0) -- (5,-0.4) -- cycle;
    \node[] at (5, 0) {Aggregation}

    % Attribute
    \coordinate (C) at (7,0);
    \draw (C) circle (0.1cm);
    \draw[shift=(C)] (0:0.1cm) -- (0:0.6cm);
    \node[] at ($(C)+(0.15,-0.3)$) {Attribute};
\end{tikzpicture}
ER diagrams are value oriented.\\
\textbf{Entities} can take on attributes. All entities in the same set have the same attributes but can have different values. Entities are identified by their values.\\
\textbf{Relationships} associate 2 or more entities from 1 or more sets. Relationship can have attributes but are identified by the participating entities. A relationhship linking $>2$ entity sets is a n-ary relation.\\
\textbf{Aggregations} are relations that are also entities of other relations.\\
\begin{tikzpicture}
    % Single Key
    \coordinate (C1) at (0,0);
    \fill[black] (C1) circle (0.1cm);
    \draw[shift=(C1)] (0:0.1cm) -- (0:0.6cm);
    \node[] at ($(C1)+(0.15,-0.3)$) {1 Attr Key};

    % Multi Key
    \coordinate (C2) at (2,0.3);
    \coordinate (C3) at ($(C2) + (0.5, 0)$);
    \coordinate (C4) at ($(C2) + (0, -0.3)$);
    \coordinate (C5) at ($(C3) + (0, -0.3)$);
    \fill[black] (C2) circle (0.1cm);
    \fill[black] (C3) circle (0.1cm);
    \draw (C4) circle (0.1cm);
    \draw (C5) circle (0.1cm);
    \draw ($(C2) + (0:0.1cm)$) -- ($(C3) + (180:0.1cm)$);
    \draw ($(C2) + (270:0.1cm)$) -- ($(C4) + (90:0.1cm)$);
    \draw ($(C3) + (270:0.1cm)$) -- ($(C5) + (90:0.1cm)$);
    \node[] at ($(C2)+(0.25,-0.6)$) {Multi Attr Key};

    % Partial Key
    \coordinate (C6) at (4,0);
    \coordinate (C7) at ($(C6) + (0.3, 0)$);
    \coordinate (E1) at ($(C7) + (0.7, 0)$);
    \coordinate (R1) at ($(E1) + (1, 0)$);
    \coordinate (C8) at ($(R1) + (0.5, 0)$);
    \coordinate (E2) at ($(C8) + (0.7, 0)$);

    % Draw shapes
    \draw (C6) circle (0.1cm);
    \fill[black] (C7) circle (0.1cm);
    \node[draw, rectangle, minimum width=0.8cm, minimum height=0.4cm] at (E1) {E1};
    \node[draw, diamond, aspect=2] at (R1);
    \fill[black] (C8) circle (0.1cm);
    \node[draw, rectangle, minimum width=0.8cm, minimum height=0.4cm] at (E2) {E2};

    % Connect lines
    \draw ($(C6) + (0:0.1cm)$) -- ($(C7) + (180:0.1cm)$);
    \draw ($(C7) + (0:0.1cm)$) -- ($(E1) + (-0.4cm, 0)$);
    \draw ($(E1) + (0.4cm, 0)$) -- ($(R1) + (-0.3cm, 0)$);
    \draw ($(R1) + (0.3cm, 0)$) -- ($(E2) + (-0.4cm, 0)$);
    \draw (C7) -- ($(C7) + (0, 0.3)$) -- ($(C8) + (0, 0.3)$) -- (C8);

    %Label
    \node[] at ($(E1)+(0.7, -0.3)$) {Partial Key}
\end{tikzpicture}
\\A Key uniquely identifies an entity within its set.\\
A Partial Key identifies an entity within the scope a relationship with another set: E1 Weak to E2(Dominant).\\
\begin{tikzpicture}
    \coordinate (E1) at (0, 0);
    \coordinate (R1) at ($(E1) + (1, 0)$);
    \coordinate (E2) at ($(R1) + (1, 0)$);
    \node[draw, rectangle, minimum width=0.8cm, minimum height=0.4cm] at (E1) {E1};
    \node[draw, diamond, aspect=2] at (R1);
    \node[draw, rectangle, minimum width=0.8cm, minimum height=0.4cm] at (E2) {E2};
    \draw ($(E1) + (0.4cm, 0)$) -- ($(R1) + (-0.3cm, 0)$);
    \draw ($(R1) + (0.3cm, 0)$) -- ($(E2) + (-0.4cm, 0)$);
    \node[font=\scriptsize] at ($(E1)+(0.65, 0.2)$) {C1};
    \node[font=\scriptsize] at ($(E2)+(-0.65, 0.2)$) {C2};
\end{tikzpicture}
C1 indicates E1's cardinality with E2 and C2 indicates E2's cardinality with E1 in the context of the relationship.\\
Cardinality is written as (min participation, max participation).\\
(1, x) mandatory part; (0, x) optional part; (x, 1) one-to-one or one-to-many; (x, n) many to many.\\
Weak entities must have (1, 1) cardinality with their dominant entity.
\underline{ER to Schema Translation Rules}:
\begin{enumerate}[leftmargin=*]
    \item Map value sets to meaningful domains.
    \item Map entity sets to relations, with candidate keys set to primary key or unique not null.
    \item Map relationship sets to relations, with the keys being the keys of participating entities
\end{enumerate}
\underline{ER to Schema Translation Exceptions}:
\begin{enumerate}[leftmargin=*]
    \item One-to-many relation sets should have the primary key be made up of  just the One side to enforce (x, 1) cardinality.
    \item  (1, 1) cardinality entity tables should be merged with the relation tables to enforce mandatory participation.
    \item Weak entity sets should be merged as in 2., and the dominant set's key should be included in the primary key of the weak set to uniquely identify rows.\\
\end{enumerate}

\subsubsection{Nested Queries}
\textbf{Table Copies}: \texttt{CREATE [TEMPORARY] <copy name> AS SELECT...} Temporary tables last until the end of the database session. Changes in the copy do not update the base table and vice versa.\\
\textbf{Views}: \texttt{CREATE VIEW <view name> AS SELECT...} Unmaterialized and updates with base table changes.\\
\textbf{Common Table Expression(CTE)}: \texttt{WITH <cte name> AS (subquery) SELECT...} Subquery only valid within the query.\\
\textbf{From}: \texttt{SELECT ... FROM (subquery) AS <from table> ...}\\ 
\textbf{Scalar Subquery}: \texttt{SELECT (subquery)} subquery can only return 1 column and 1 row.\\
\textbf{In}: \texttt{WHERE ... IN (subquery)} checks for match in subquery.\\
\textbf{Exists}: \texttt{WHERE ... EXISTS (subquery)} true if \texttt{|subquery| > 0}.\\
\textbf{Any/All}: \texttt{WHERE ... <comparator> ANY/ALL (subquery)} returns true if comparison is true for any/all values in subquery. \texttt{IN} $\equiv$ \texttt{=ANY}. \texttt{ALL} adds expressiveness.\\
\textbf{Correlated Subqueries} depend on the values of the outer query for their query. All subqueries can be correlated.\\
\textbf{Lexical Scoping}: columns from outer table can be used in inner query but not vice versa.\\
All checks above can be negated with \texttt{NOT} and also be used with aggregate functions in the \texttt{HAVING} clause.\\

\subsubsection{Relational Algebra}
\textbf{Selection}(filtering) $\sigma_{[cond]}(R)$ Conditions use $=,\neq,\lt,\gt,\leq,\geq,\land,\lor,\neg$; \\
\textbf{Projection}(select column) $\pi_{[cols]}(R)$ or $\pi[R.col]$; \\
\textbf{Renaming} $\rho(R1, R2(A1\rightarrow B1))$; \\
\textbf{Union} $\cup$; \textbf{Intersection} $\cap$; \textbf{Set Difference} $-$; \\
\textbf{Cross Product} $\times$; \textbf{Inner/Natural Join} $\bowtie$;\\
\textbf{Outer Joins} $\leftouterjoin, \rightouterjoin, \fullouterjoin$; Assignment $:=$.

\subsubsection{SQL From Code}
SQL Execution from Code: Declare Statements and Arguments $\rightarrow$ Connect to Database $\rightarrow$ Execute Queries $\rightarrow$ Deallocate Resources.\\
\textbf{Statement Level Interface}: Use preprocessor to parse embedded SQL statements  into executable code.\\
\textbf{Prepared Statements} allow frequently used statements to be compiled by the database only once. Manual deallocation is required.\\
\textbf{Call Level Interface}: Execute SQL statement strings via API calls.\\ 

\subsubsection{Functions and Procedures}
Procedure Languages allows implementation of complex logic, code reuse and easy maintenance.\\
\textbf{Function}: \texttt{CREATE OR REPLACE FUNCTION <func name> ([<param> <type>]) RETURNS <type> AS \$\$ ... \$\$ LANGUAGE <language>}. Returns a value(\texttt{VOID} if nth) but is not a transaction.\\
Return type can be table name if return table record.  \texttt{RECORD} and multiple \texttt{OUT} parameters or \texttt{TABLE([<name> <type>]+)} can be used for custom records. \texttt{SETOF} is used for multiple records. \\
\textbf{Procedure}: \texttt{CREATE OR REPLACE PROCEDURE <proc name> ([IN/OUT <param> <type>]) AS \$\$ ... \$\$ LANGUAGE <language>}. No return value but can set an out parameter. Is a transaction that can be rolled back.\\

\subsubsection{Triggers}
\texttt{CREATE [CONSTRAINT] TRIGGER <name> BEFORE/INSTEAD OF/AFTER INSERT/UPDATE[ OF <col>]/DELETE ON <table> [DEFFERABLE INITTIALLY DEFERRED/IMMEDIATE] FOR EACH ROW[ WHEN <condition>]/STATEMENT EXECUTE FUNCTION <trigger func>;}
Trigger function has no params and returns type \texttt{TRIGGER}.\\
\textbf{Row} level triggers have access to \texttt{OLD} and/or \texttt{NEW} variables, the row before/after the operation, based on which operation is performed. Trying to access \texttt{OLD} or \texttt{NEW} in a statement level trigger will result in an error. \\
Return \texttt{NULL} on \textbf{Before Row} level triggers, use \texttt{RAISE EXCEPTION} on \textbf{After Row} and \textbf{Statement} level triggers to abort the transaction.\\
\texttt{INSTEAD OF} triggers on non aggregate views can transfer a \textbf{Row} level operation directly to the underlying table. \\
\texttt{WHEN} cannot use subquery, not be used with \texttt{INSTEAD OF} and access invalid \texttt{NEW}/\texttt{OLD} variables.\\
\textbf{Deferred Triggers}: defer constraint checks to the end of a transaction, only works on \texttt{CONSTRAINT TRIGGERS} with \texttt{DEFERRABLE INITTIALLY DEFERRED/IMMEDIATE} set. Set defer status in transaction with \texttt{SET CONSTRAINTS <trigger name> DEFERRED/IMMEDIATE;}\\
\textbf{Order of Activation}: \texttt{BEFORE STATEMENT} $\rightarrow$ \texttt{BEFORE ROW} $\rightarrow$ \texttt{AFTER ROW} $\rightarrow$ \texttt{AFTER STATEMENT} $\rightarrow$ then alphabetic order. \texttt{NULL} on a \texttt{BEFORE ROW} trigger omits subsequent \texttt{BEFORE ROW} triggers.\\

\subsubsection{PLPGSQL}
\textbf{Structure}: \texttt{DECLARE <variable dec> BEGIN <logic> END;} \\
\textbf{Assignment}: \texttt{<var> := <expression>;}
\textbf{Selection}: \texttt{IF... THEN... ELSIF... THEN... ELSE... ENDIF}\\
\textbf{Repetition}: \texttt{LOOP EXIT WHEN ... END LOOP}, \texttt{WHILE... / FOR <row> IN <query> LOOP ... END LOOP}. \texttt{RETURN NEXT} is used to yield the next value or record and \texttt{RETURN QUERY} the next record, followed by a final \texttt{RETURN} with no value when returning a set of values.\\
\textbf{Cursor}: Allows access to each row of a \texttt{SELECT} statement.\\
\texttt{DECLARE} cursor \texttt{<cursor> CURSOR FOR <query>;}\\
\texttt{FETCH [PRIOR/ FIRST/ LAST/ABSOLUTE <num>] FROM <cursor> INTO <record>;}\\
\texttt{EXIT WHEN NOT FOUND;} \texttt{CLOSE <cursor>;}\\

\noindent\textbf{SQL Injection}: Attacks by using \texttt{;} or \texttt{--} to add or remove parts of the query to execute the attacker's query. Use prepared statements so user input is treated as a string and can't change the control flow.\\

\subsubsection{Functional Dependencies(FD)}
\textbf{Redundancies} can lead to \textbf{anomalies} if not all copies are changed.\\
\textbf{Normalizing} tables reduces redundancies and anomalies.\\
B is a FD on A ($A\rightarrow B$) $\Leftrightarrow$ A uniquely determines B.\\
\textbf{Armstrong's Axioms}: \underline{Reflexivity}, subset is always FD on superset; \underline{Augmentation}, if $A\rightarrow B$ then $AC\rightarrow BC$; \underline{Transitivity}, $A\rightarrow B$ \& $B\rightarrow C$ $\Rightarrow$ $A\rightarrow C$.\\
\textbf{Additional Rules}: \underline{Decomposition}, $A\rightarrow BC$ $\Rightarrow$ $A\rightarrow B$ \& $A\rightarrow C$; \underline{Union},  $A\rightarrow B$ \& $A\rightarrow C$ $\Rightarrow$ $A\rightarrow BC$.\\
\textbf{Closure} of S, $\{S\}^+$, is all the attributes uniquely determined by S. Compute by iteratively adding FD's until no more.\\
\textbf{Superkeys} decide(have closures of) all other attributes.\\
\textbf{Keys} are minimal closures or superkeys, no subset that is also a key.\\
If A is not FD on any other attribute, it must be in the key.\\
\textbf{Prime attributes} appear in at least 1 key.\\ 
\underline{Find keys of S}: 
\begin{enumerate}[leftmargin=*]
    \item Compute closures of all subsets of S .
    \item Select superkeys with no subset that is a superkey.\\
\end{enumerate}

\subsubsection{Boyce-Codd Normal Form(BCNF)}
For $A\rightarrow B$, \textbf{Trivial} FD if $B\subseteq A$; \textbf{Non-Trivial} FD if $B \cap A \neq \varnothing$ and $B \nsubseteq A$; \textbf{Completely/Decomposed Non-Trivial(DNT)} if $B \cap A = \varnothing$.\\
To find DNT FDs: Find closures of all subsets, remove trivial attributes, then decompose FDs.\\
BCNF if every DNT FD is on a superkey. No dependencies on non-superkeys.\\
\underline{Check if S is in BCNF}: 
\begin{enumerate}[leftmargin=*]
    \item Derive keys and DNT FDs.
    \item Check if every DNT FD is on a superkey.
\end{enumerate}
\underline{Optimized}: Check if a attribute subset has a more but not all closure.\\
\underline{Decompose S into BCNF}: 
\begin{enumerate}[leftmargin=*]
    \item Identify a more but not all closure subset $X$ in $S$.
    \item   Split table into $S_1=\{X\}^+$ \& $S_2=\{X, $$R-\{X\}^+\}$.
    \item  Recurse if $S_1$ or $S_2$ is not BCNF. 2 attr table is always BCNF.
\end{enumerate}
\underline{Project FDs to compute sub-table closure}
\begin{enumerate}[leftmargin=*]
    \item Derive closures of every attribute subset of $S_1$/$S_2$ on $S$.
    \item Remove attributes on the RHS that are not in $S_1$/$S_2$.
\end{enumerate}
Multiple BCNF decompositions may exist.\\
\textbf{Lossless-Join Decomposition}(LJD) allows the original table to be reconstructed from the decomposed tables without data loss. \underline{Criteria}:
\begin{enumerate}[leftmargin=*]
    \item $R1\cup R2=R$
    \item $R1\cap R2\neq \varnothing$
    \item $R1\cap R2$ is a candidate key for at least 1 of $R1$ or $R2$. 
\end{enumerate}
Decomposition algorithm above maintains \textbf{Binary LJD}.\\
However, FDs that are split into separate tables are not preserved.\\
FDs are \underline{preserved} if every FD in the original table(S) can be derived from FDs in the decomposed table(S') and vice versa. aka S is \underline{equivalent} to S'.\\

\subsubsection{3rd Normal Form(3NF)}
3NF iff for every DNT FD, the LHS is a superkey or the RHS is a prime attribute. BCNF $\subset$ 3NF. 3NF preserves all FDs.\\
\underline{Check if S is in 3NF}: 
\begin{enumerate}[leftmargin=*]
    \item Derive keys and DNT FDs of R.
    \item For each DNT FD check if it satisfies above condition.
\end{enumerate}
A \underline{minimal basis/cover}(MB) of S is:
\begin{enumerate}[leftmargin=*]
    \item Every FD in MB can be derived from S.
    \item Every FD in MB is DNT.
    \item No FD in MB is redundant, can be derived from another FD.
    \item LHS of every FD is not redundant, removing an attribute on the LHS results in a new FD that cannot be derived from original FDs.
\end{enumerate}
\underline{Find minimal basis of S}:
\begin{enumerate}[leftmargin=*]
    \item Decompose FDs so RHS only has 1 attribute.
    \item Remove redundant attributes on LHS of FDs, aka the reduced LHS is implied by existing relations, aka does the cover of the reduced LHS include the RHS.
    \item Remove redundant FDs. The cover of the remaining FDs covers the redundant FD.
\end{enumerate}
\underline{Decompose S into 3NF}:
\begin{enumerate}[leftmargin=*]
    \item Find a minimal basis of S.
    \item Combine FDs with same LHS by Union Rule. 
    \item Create a table for each remaining FD.
    \item If no table contains a key of S, create a new table with a key of S.
    \item Remove subsumed tables, a table that is a subset of another.
\end{enumerate}
Step 4 is needed to ensure \textbf{LJD}.

\end{multicols*}